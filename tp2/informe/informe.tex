% ------ headers globales -------------
\documentclass[11pt, a4paper, twoside]{article}
\usepackage{header}
\usepackage{config}
% -------------------------------------
\begin{document}

% -- Carátula --
\clearpage{\pagestyle{empty}% parametros para la caratula (caratula.sty)

\materia{Sistemas Operativos}
%\submateria{}
\titulo{Trabajo Práctico 3}
\subtitulo{Algoritmos en sistemas distribuidos}
%\subtitulo{Escape en Sistemas}
\fecha{13 de noviembre de 2014}
\integrante{Rodriguez, Pedro}{197/12}{pedro3110.jim@gmail.com}
\integrante{Benegas, Gonzalo Segundo}{958/12}{gsbenegas@gmail.com}
\integrante{Barrios, Leandro Ezequiel}{404/11}{ezequiel.barrios@gmail.com}
%\grupo{Grupo ??}

\maketitle
}

%-- Índice --
\clearpage{%
  \pagestyle{empty}\tableofcontents%
  \vspace{3cm}%
  \cleardoublepage%
}
%-- A partir de aquí, pongo el contador de páginas en 1 --
\setcounter{page}{1}

\index{Introducción}
\section{Introducción}
En el presente trabajo práctico empleamos técnicas de programación paralela. En particular, nos centramos
en el diseño y la implementación de un servidor que a través del uso de múltiples hilos de ejecución
(\texttt{multithreading}) se encarga de atender concurrentemente las peticiones a distintos clientes. Esto
lo logramos siguiendo el modelo de $workers$ propuesto por la cátedra en el enunciado del TP.

En el TP se nos propone resolver el problema de coordinar adecuadamente la evacuación de un edificio. En el
modelo provisto por la cátedra, contamos con un sistema que procesa los pedidos de los clientes de forma
secuencial. Y nuestro objetivo es usar técnicas de multiprogramación para procesar estos pedidos de forma
paralela. En nuestro modelo, los clientes son personas que están en un espacio rectangular dividido en
metros cuadrados. Cada persona está en algún metro cuadrado y cada baldosa tiene un límite predefinido en
cuanto a la cantidad de personas que pueden estar en ese espacio al mismo tiempo.

\newpage
\index{Desarrollo}
\section{Desarrollo}
Para implementar el \texttt{servermulti} nos basamos en el código de \texttt{servermono} provisto por la
cátedra. Las tareas que tuvimos que realizar para lograr la administración paralela de los pedidos por
parte de los clientes fueron:
\begin{enumerate}
\item para cada cliente que pida entrar a nuestro sistema, largamos un $thread$ que se encargue de la
  comunicación entre \texttt{servermulti} y dicho cliente. Todos estos $threads$ comparten el acceso a un
  mismo bloque de memoria. Es esta la razón por la cuál tuvimos que emplear \texttt{mutex} y \texttt{locks}
  para administrar correctamente el acceso a estas posiciones de memoria.
\item cada vez que un cliente pide desplazarse de una casilla del aula a otra, manejar correctamente el
  acceso a dichas celdas y actualizar correctamente las variables globales que determinan el estado actual
  del sistema en cada instante
\item manejar correctamente la salida definitiva del edificio de las personas que ya están afuera del mismo.
  Para esto tuvimos que tener cuidado con,  para cada persona que logra salir de la habitación, determinar
  si hay algún rescatista libre en ese momento para ponerle la máscara y así poder posteriormente (cuando
  hayan llegado otras cuatro personas junto a ella) salir definitivamente del edificio y salvarse. Si no lo
  hay, la persona tiene que esperar a que llegue un rescatista y le ponga una máscara antes de poder salvarse
  definitivamente (podrían haber más de cinco personas fuera del aula pero esperando a que lleguen rescatistas
  para que le pongan la máscara y recién ahí, cuando hayan cinco personas con máscara, puedan salir).

\end{enumerate}

\par Un problema importante que nos surgió fue el de evitar $deadlock$ cuando dos $a$ y $b$ que están en las
casillas $A$ y $B$ piden acceso a $B$ y $A$ respectivamente. Para arreglar este potencial problema, lo que
hicimos fue...

\par Para manejar correctamente el tema de los rescatistas utilizamos dos variables: una para contabilizar
la cantidad de personas fuera del edificio \textbf{con máscara} y por otro lado la cantidad de personas
fuer del edificio pero \textbf{esperando máscara}.

\par Cuando testeamos nuestro código, vimos que el tiempo que tardaban los rescatistas en ponerle la máscara
a cada una de las personas era muy corto. Entonces, para que la existencia de rescatistas tenga sentido y su
existencia sea considerable cuando corremos nuestro \texttt{servermulti}, agregamos un tiempo de tardanza
determinado para que cada rescatista tarde en colocar una máscara a una persona.

\newpage
\index{Conclusión}
\section{Conclusión}

\end{document}
