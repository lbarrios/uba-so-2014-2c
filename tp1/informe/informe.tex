\documentclass[a4paper]{article}

\usepackage[utf8]{inputenc}
\usepackage{amsmath}
\usepackage{graphicx}
\usepackage[colorinlistoftodos]{todonotes}

\usepackage{multicol}
\usepackage{makeidx}
\usepackage{hyperref}
\usepackage{caption}
\usepackage{amsfonts}
\usepackage{amssymb}
\usepackage[utf8]{inputenc}
\usepackage{verbatim}
\usepackage{listings}
\lstset{language=C++, showstringspaces=false, tabsize=2, breaklines=true, title=\lstname}

\usepackage[margin=0.5in]{geometry}

\title{TP2 Metnum}

\author{Pedro Rodriguez, Gonzalo Benegas, Leandro Ezequiel Barrios}

\date{\today}

\makeindex

\begin{document}
\newgeometry{margin=2cm}
\pagenumbering{gobble}
\raggedleft
\includegraphics[width=8cm]{logo1.jpg}\\

\raggedright
\vspace{3cm}
{\Huge \bfseries Trabajo Práctico 1}
\rule{\textwidth}{0.02in}
\large Martes 16 de Septiembre de 2014 \hfill Sistemas Operativos
\vspace{1.5cm}

 
\centering \LARGE Scheduling 
\vspace{1.5cm}

\normalsize
\begin{tabular}{|l@{\hspace{4ex}}c@{\hspace{4ex}}l|}
        \hline
        \rule{0pt}{1.2em}Integrante & LU & Correo electr\'onico\\[0.2em]
        \hline
        \rule{0pt}{1.2em}  Pedro Rodriguez &197/12 &\tt pedro3110.jim@gmail.com \\[0.2em]
        \rule{0pt}{1.2em}  Gonzalo Segundo Benegas &958/12 &\tt  gsbenegas@gmail.com \\[0.2em]
        \rule{0pt}{1.2em}  Leandro Ezequiel Barrios &404/11 &\tt ezequiel.barrios@gmail.com \\[0.2em]
        \hline
\end{tabular}

\vspace{1.0cm}
\raggedright

\begin{multicols}{2}
\includegraphics[width=8cm]{logo-uba.png}

\columnbreak
\vspace*{4.5cm}
\raggedleft
\textbf{Facultad de Ciencias Exactas y Naturales}\\
Universidad de Buenos Aires\\
\small
Ciudad Universitaria - (Pabellon I/Planta Baja)\\
Intendente G\"uiraldes 2160 - C1428EGA\\
Ciudad Autonoma de Buenos Aires - Rep. Argentina\\
Tel/Fax: (54 11) 4576-3359\\
http://www.fcen.uba.ar
\end{multicols}

\restoregeometry

\clearpage

\pagenumbering{arabic}

\tableofcontents

\vspace{3cm}

\begin{abstract}
En el presente trabajo pr\'actico estudiaremos algunos de los m\'etodos m\'as com\'unmente utilizados 
en la actualidad por distintos sistemas operativos para manejar correcta y eficientemente los 
diversos procesos que se ejecutan concurrentemente en m\'aquinas con uno o m\'as procesadores. \\
Intentaremos detectar las ventajas y desventajas de cada m\'etodo, as\'i como los escenarios en los
cuales uno puede ser m\'as eficiente que otro. Para esto, dividimos el TP en tres partes: en la
primera, presentamos el simulador simusched y lo corremos para algunas tareas espec\'ificas. En la
segunda parte, extendemos el simulador con nuevos schedulers implementados por nosotros y finalmente,
en la tercera parte evaluamos y analizamos los distintos algoritmos de scheduling ya presentados. \\
\end{abstract}

\index{Parte I - Entendiendo el simulador simusched}
\section{Parte I - Entendiendo el simulador simusched}

\subsection{Ejercicio 1}
En este ejercicio, programamos la tarea TaskConsola, que simular\'ia una tarea interactiva con
el usuario. \\
Para la implementaci\'on de esta tarea, primero tuvimos que registrarla para que el simulador la 
reconozca en el archivo tasks.cpp, con la funci\'on tasksinit(void). En ese mismo archivo implementamos
la tarea, que para el proceso n\'umero pid recibe los tres par\'ametros: n, bmin y bmax. Simplemente
hacemos un ciclo que cicle n veces y que cada vez tome un entero al azar $r$ entre $bmin$ y 
$bmax$, y cada vez llamamos a la funci\'on usoIO(pid, r), que simula hacer uso de dispositivos de
entrada/salida. \\
Nota(1): se denomina llamada bloqueante a aquellas llamadas en las que, si lo que esta  
solicita no est\'a disponible, entonces el proceso llamador se queda bloqueado a la espera de un
resultado. Es decir, no permite que otros procesos dependientes de este sigan ejecutando. Por 
el contrario, en una llamada no bloqueante, si el proceso llamador no encuentra lo que estaba
buscando, el proceso devuelve de todas formas alg\'un resultado, permitiendo que otros procesos
sigan ejecutando sin tener que esperar a que este proceso llamador reciba el resultado que est\'a
esperando. \\
Nota(2): para elegir el n\'umero pseudo-aleatorio hacemos uso de la funci\'on rand() provista
por la librer\'a standard de C. \\

\subsection{Ejercicio 2}
Para seguir entendiendo el funcionamiento del simulador simusched, ahora pasamos a escribir un lote
de 3 tareas distintas: una intensiva en CPU y las otras dos de tipo interactivo. Vamos a ejecutar y
graficar la simulaci\'on usando el algoritmo FCFS para 1, 2 y 3 n\'ucleos. \\
Pudimos observar que ...


\index{Parte II - Extendiendo el simulador con nuevos schedulers}
\section{Parte II - Extendiendo el simulador con nuevos schedulers}

\subsection{Ejercicio 3}
\subsection{Ejercicio 4}
\subsection{Ejercicio 5}
En este ejercicio, basados en el paper de Waldspurger, C.A. y Weihl en el cu\'al presentan un 
novedoso sistema de Scheduling llamado $Lottery Scheduling$, implementamos una clase que llamamos
SchedLottery que recibe como par\'ametros el quantum que se va a estar simulando y una semilla 
pseudo-aleatoria, que el scheduler va a utilizar para el manejo de los procesos. Como dice en el
enunciado, b\'asicamente nos interes\'o implementar la idea b\'asica del algoritmo y la 
optimizaci\'on de tickets compensatorios. \\
La clase SchedLottery tiene como parte p\'ublica las funciones $load(int pid)$, $unblock(int pid)$ 
y $tick(cpu, Motivo)$. \\
i) La funci\'on load la utilizamos para notificar al scheduler
que un nuevo proceso ha llegado. Cada vez que llega un nuevo proceso le damos un ticket. \\
ii) La funci\'on
$tick(cpu, motivo)$. El par\'ametro motivo puede significar que una de tres cosas
ocurrieron durante el \'ultimo ciclo del reloj: la tarea pid consumi\'o todo su
ciclo utilizando el CPU (en cuyo caso incrementamos la cantidad de ticks y si esta cantidad supera el
quantum, la desalojamos), la tarea ejecut\'o una llamada bloqueante o permaneci\'o 
bloqueada durante el \'ultimo ciclo (en cuyo caso proveemos al proceso bloqueado la compensaci\'on 
de tickets correspondiente) \'o porque la tarea pid termin\'o (en cuyo caso desalojamos la tarea
actual). Despu\'es de hacer estos chequeos, corremos $run\_lottery$ que se encarga de efectuar el 
sorteo de tickets para la pr\'oxima vez que se necesite elegir un proceso ganador para asignarle 
los recursos necesarios. \\
iii) La funci\'on unblock(pid) hace que en la pr\'oxima llamada a la funci\'on tick, el proceso pid 
est\'e disponible para ejecutar. Lo que hacemos, es entregarle al proceso pid la cantidad de tickets
que ten\'ia el proceso antes de ser bloqueado multiplicado por la compensaci\'on correspondiente. \\
Por otro lado, la clase SchedLottery tiene una parte privada, que consta de variables como $quantum$, 
que usamos para saber el quantum que tiene disponible cada proceso para ejecutarse, $seed$, que
usamos como semilla para elegir el ticket ganador cada vez que hagamos un sorteo, $vector<ticket>$, que
usamos para saber qu\'e tickets corresponden a cada proceso, $totaltickets$ para tener r\'apido acceso
a la cantidad de tickets que hay distribuidos entre los procesos y $ticketnumber$ que usamos para 
llevar la cuenta de cu\'antas veces llamamos a la funci\'on tick(cpu,motivo). Adem\'as, contamos con 
las funciones privadas compensa(int), desaloja(int) y ticketsindex(int). La primera se encarga de
calcular y asignar la compensaci\'on deseada a un proceso cuando corresponda; la segunda de que cuando
se bloquea un proceso, no est\'e disponible para ejecutar en la pr\'oxima ejecuci\'on de la funci\'on
tick(cpu,motivo). Es decir, que la probabilidad de ser elegible en el sorteo sobre los tickets sea cero.
Para eso, lo que hicimos fue sacarle todos los tickets temporalmente a dicho proceso. \\ 




\index{Parte III - Evaluando los algoritmos de scheduling}
\section{Parte III - Evaluando los algoritmos de scheduling}

\subsection{Ejercicio 6}
\subsection{Ejercicio 7}
\subsection{Ejercicio 8}
\subsection{Ejercicio 9}
\subsection{Ejercicio 10}














\end{document}
