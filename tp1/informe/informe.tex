\documentclass[a4paper]{article}

\usepackage[utf8]{inputenc}
\usepackage{amsmath}
\usepackage{graphicx}
\usepackage{float}
%\usepackage[colorinlistoftodos]{todonotes}

\usepackage{multicol}
\usepackage{makeidx}
\usepackage{hyperref}
\usepackage{caption}
\usepackage{amsfonts}
\usepackage{amssymb}
\usepackage[utf8]{inputenc}
\usepackage{verbatim}
\usepackage{listings}
\lstset{language=C++, showstringspaces=false, tabsize=2, breaklines=true, title=\lstname}

\usepackage[margin=0.5in]{geometry}

\title{TP2 Metnum}

\author{Pedro Rodriguez, Gonzalo Benegas, Leandro Ezequiel Barrios}

\date{\today}

\makeindex

\begin{document}
\newgeometry{margin=2cm}
\pagenumbering{gobble}
\raggedleft
\includegraphics[width=8cm]{logo1.jpg}\\

\raggedright
\vspace{3cm}
{\Huge \bfseries Trabajo Práctico 1}
\rule{\textwidth}{0.02in}
\large Martes 16 de Septiembre de 2014 \hfill Sistemas Operativos
\vspace{1.5cm}

 
\centering \LARGE Scheduling 
\vspace{1.5cm}

\normalsize
\begin{tabular}{|l@{\hspace{4ex}}c@{\hspace{4ex}}l|}
        \hline
        \rule{0pt}{1.2em}Integrante & LU & Correo electr\'onico\\[0.2em]
        \hline
        \rule{0pt}{1.2em}  Pedro Rodriguez &197/12 &\tt pedro3110.jim@gmail.com \\[0.2em]
        \rule{0pt}{1.2em}  Gonzalo Segundo Benegas &958/12 &\tt  gsbenegas@gmail.com \\[0.2em]
        \rule{0pt}{1.2em}  Leandro Ezequiel Barrios &404/11 &\tt ezequiel.barrios@gmail.com \\[0.2em]
        \hline
\end{tabular}

\vspace{1.0cm}
\raggedright

\begin{multicols}{2}
\includegraphics[width=8cm]{logo-uba.png}

\columnbreak
\vspace*{4.5cm}
\raggedleft
\textbf{Facultad de Ciencias Exactas y Naturales}\\
Universidad de Buenos Aires\\
\small
Ciudad Universitaria - (Pabellon I/Planta Baja)\\
Intendente G\"uiraldes 2160 - C1428EGA\\
Ciudad Autonoma de Buenos Aires - Rep. Argentina\\
Tel/Fax: (54 11) 4576-3359\\
http://www.fcen.uba.ar
\end{multicols}

\restoregeometry

\clearpage

\pagenumbering{arabic}

\tableofcontents

\vspace{3cm}

\begin{abstract}
En el presente trabajo pr\'actico estudiaremos algunos de los m\'etodos m\'as com\'unmente utilizados 
en la actualidad por distintos sistemas operativos para manejar correcta y eficientemente los 
diversos procesos que se ejecutan concurrentemente en m\'aquinas con uno o m\'as procesadores. \\
Intentaremos detectar las ventajas y desventajas de cada m\'etodo, as\'i como los escenarios en los
cuales uno puede ser m\'as eficiente que otro. Para esto, dividimos el TP en tres partes: en la
primera, presentamos el simulador simusched y lo corremos para algunas tareas espec\'ificas. En la
segunda parte, extendemos el simulador con nuevos schedulers implementados por nosotros y finalmente,
en la tercera parte evaluamos y analizamos los distintos algoritmos de scheduling ya presentados. \\
\end{abstract}

\newpage
\index{Parte I - Entendiendo el simulador simusched}
\section{Parte I - Entendiendo el simulador simusched}

\subsection{Ejercicio 1}
En este ejercicio, programamos la tarea TaskConsola, que simular\'ia una tarea interactiva con
el usuario. \\
Para la implementaci\'on de esta tarea, primero tuvimos que registrarla para que el simulador la 
reconozca en el archivo tasks.cpp, con la funci\'on tasksinit(void). En ese mismo archivo implementamos
la tarea, que para el proceso n\'umero pid recibe los tres par\'ametros: n, bmin y bmax. Simplemente
hacemos un ciclo que cicle n veces y que cada vez tome un entero al azar $r$ entre $bmin$ y 
$bmax$, y cada vez llamamos a la funci\'on usoIO(pid, r), que simula hacer uso de dispositivos de
entrada/salida. \\
Nota(1): se denomina llamada bloqueante a aquellas llamadas en las que, si lo que esta  
solicita no est\'a disponible, entonces el proceso llamador se queda bloqueado a la espera de un
resultado. Es decir, no permite que otros procesos dependientes de este sigan ejecutando. Por 
el contrario, en una llamada no bloqueante, si el proceso llamador no encuentra lo que estaba
buscando, el proceso devuelve de todas formas alg\'un resultado, permitiendo que otros procesos
sigan ejecutando sin tener que esperar a que este proceso llamador reciba el resultado que est\'a
esperando. \\
Nota(2): para elegir el n\'umero pseudo-aleatorio hacemos uso de la funci\'on rand() provista
por la librer\'a standard de C. \\

\newpage
\subsection{Ejercicio 2}
Para seguir entendiendo el funcionamiento del simulador simusched, ahora pasamos a escribir un lote
de 3 tareas distintas: una intensiva en CPU y las otras dos de tipo interactivo. Vamos a ejecutar y
graficar la simulaci\'on usando el algoritmo FCFS para 1, 2 y 3 n\'ucleos. \\
En el caso de la tarea que hace uso intensivo del cpu, pudimos observar que usando dos cpu's, las
tareas se logran ejecutar pr\'acticamente en la mitad de tiempo (en el gr\'afico se ve claramente como
se aprovecha el hecho de que hay dos cpu's, y en todo momento se puede observar c\'omo hay dos cpu's 
trabajando al mismo tiempo en dos tareas distintas. Cuando introdujimos un tercer cpu, sucedi\'o 
exactamente lo mismo: cada vez que alg\'un cpu terminaba de ejecutar un proceso, inmediatamente
comenzaba a ejecutar el pr\'oximo proceso a\'un no atendido. \\
En el caso de las tareas interactivas, pudimos observar los momentos en los que se detectaba que 
alg\'un procesador era bloqueado y c\'omo en ese tiempo el procesador se quedaba trabado en la misma 
tarea. Creemos que ese tiempo de espera podr\'ia ser aprovechado para seguir ejecutando alg\'un otro
proceso, implementando alg\'un otro sistema de scheduling. \\
Otra cosa importante a destacar, es que en los tres lotes de tareas, cuando pasamos de usar un \'unico
core a usar dos, el tiempo total que se tard\'o en ejecutar todas las tareas disminuy\'o a la mitad.
Mientras que cuando usamos tres cores, el tiempo no disminuy\'o a un tercio de lo que tardaba antes,
que es lo que nosotros esper\'abamos. Esto nos da la pauta de que m\'as procesadores no siempre
mucha m\'as eficiencia y velocidad: hay otros factores que son m\'as importantes, como el algoritmo de
scheduling utilizado, y la consideraci\'on de las caracter\'isticas del lote de tareas que voy a querer
correr. \\
A continuaci\'on, presentamos los diagramas de Gantt de los cuales hablamos: \\

\includegraphics [width=17cm]{../graficos/sched_fcfs/cpu_intensivo.png} {Fig. 2.1}
\includegraphics [width=17cm]{../graficos/sched_fcfs/cpu_intensivo2.png} {Fig. 2.2}
\includegraphics [width=17cm]{../graficos/sched_fcfs/cpu_intensivo3.png} {Fig. 2.3}
\includegraphics [width=17cm]{../graficos/sched_fcfs/interactivo.png} {Fig. 2.4}
\includegraphics [width=17cm]{../graficos/sched_fcfs/interactivo_2.png} {Fig. 2.5}
\includegraphics [width=17cm]{../graficos/sched_fcfs/interactivo_3.png} {Fig. 2.6}
\includegraphics [width=17cm]{../graficos/sched_fcfs/interactivo2.png} {Fig. 2.7}
\includegraphics [width=17cm]{../graficos/sched_fcfs/interactivo2_2.png} {Fig. 2.8}
\includegraphics [width=17cm]{../graficos/sched_fcfs/interactivo2_3.png} {Fig. 2.9}


\newpage
\index{Parte II - Extendiendo el simulador con nuevos schedulers}
\section{Parte II - Extendiendo el simulador con nuevos schedulers}

\subsection{Ejercicio 3}
Ahora pasaremos a explicar algunos aspectos relevantes de la implementaci\'on que hicimos del shceduler
$Round-Robin$, cuya idea b\'asica es asignar espacios de tiempo equitativos a cada proceso y ir pasando
de uno a otro en forma circular, sin asignar ning\'un tipo de prioridad a ning\'un proceso. Una
caracter\'istica interesante de este sistema de scheduling es que nos aseguramos de no tener el problema
conocido como $starvation$. \\
En la implementaci\'on, recibimos como par\'ametro la cantidad de n\'ucleos con la que vamos a trabajar
y los valores de sus respectivos $quantums$. Utilizamos una \'unica cola global, para permitir as\'i la
migraci\'on de proceso entre n\'ucleos. La clase SchedRR consta de una parte p\'ublica y una privada. En 
la p\'ublica, se cuenta (al igual que en SchedLottery) con las funciones $load(pid)$, $unblock(pid)$ 
y $tick(cpu, motivo)$. La funci\'on de load es la de notificar al scheduler que un nuevo proceso ha llegado.
Cada vez que esto sucede, agregamos a la cola de procesos el pid del de dicho proceso. Unblock se encarga 
de que en la pr\'oxima llamada a la funci\'on tick, el proceso pid est\'e disponible para ejecutar. Y 
finalmente, tick se encarga de manejar, de acuerdo a qu\'e fue lo \'ultimo que hizo el proceso antes de llamar
a dicha funci\'on y a la porci\'on del $quantum$ ya utilizado por el proceso, de encolar en la cola de procesos
al proceso actual y pasar a correr el pr\'oximo proceso si ya se utiliz\'o todo el $quantum$, de pasar a correr 
el pr\'oximo proceso y dejar fuera de la lista de procesos al proceso actual si dicho proceso ha finalizado, o de
poner a correr el pr\'oximo proceso y encolar en la lista de procesos al proceso actual si dicho proceso se ha
bloqueado por alguna raz\'on. \\
Por otro lado, esta la parte privada de la clase SchedRR. Consta de algunas variables que utilizamos para efectuar
correctamente el manejo de procesos seg\'un la pl\'itica del round-robin. Usamos un $vector<int> cores\_ticks$, 
con el que llevamos la cuenta en todo momento de la cantidad de ticks que consumi\'o el proceso actual, 
para cada n\'ucleo. Tambi\'en utilizamos un $vector<int> cores\_quantums$, en el cu\'al almacenamos los quantums
de cada procesador que nos son pasados por par\'ametro. Otra estructura de datos importante es una
$queue<int> process\_queue$, en la cu\'al almacenamos todos los procesos que est\'an esperando a ser ejecutados.
En $cores_count$ guardamos la cantidad de n\'ucleos con la que vamos a trabajar. Finalmente, implementamos la
funci\'on $run\_next(cpu)$, que devuelve el pid del pr\'oximo proceso de la cola a ejecutar. S\'olo la invocamos
cuando es necesario pasar a ejecutar un nuevo proceso. Y lo que hacemos es sacar el primer proceso que est\'a 
esperando en la cola de la misma y ponerlo a correr, al mismo tiempo que inicializamos la variable
$cant\_ticks[pid]$ a cero (inicializamos el contador de ticks para el proceso actual en ejecuci\'on a cero). \\

\newpage
\subsection{Ejercicio 4}

\newpage
\subsection{Ejercicio 5}
En este ejercicio, basados en el paper de Waldspurger, C.A. y Weihl en el cu\'al presentan un 
novedoso sistema de Scheduling llamado $Lottery Scheduling$, implementamos una clase que llamamos
SchedLottery que recibe como par\'ametros el quantum que se va simular y una semilla 
pseudo-aleatoria, que el scheduler va a utilizar para el manejo de los procesos. Como dice en el
enunciado, b\'asicamente nos interes\'o implementar la idea b\'asica del algoritmo y la 
optimizaci\'on de tickets compensatorios y en este caso, siempre trabajamos c\'olo con un n\'ucleo. \\
La clase SchedLottery tiene como parte p\'ublica las funciones $load(pid)$, $unblock(pid)$ 
y $tick(cpu, motivo)$. \\
i) La funci\'on load(pid) la utilizamos para notificar al scheduler
que un nuevo proceso ha llegado. Cada vez que llega un nuevo proceso le damos un ticket a dicho proceso.
De esta forma, en el pr\'oximo sorteo el proceso ser\'a candidato a ganarlo y as\'i ser el pr\'oximo 
en ser ejecutado. \\
ii) La funci\'on
tick(cpu, motivo). El par\'ametro motivo puede significar que una de tres cosas
ocurrieron durante el \'ultimo ciclo del reloj: la tarea pid consumi\'o todo su
ciclo utilizando el CPU n\'umero $cpu$ (en cuyo caso incrementamos la cantidad de ticks y si esta cantidad supera el
quantum de dicho procesador, la desalojamos), la tarea ejecut\'o una llamada bloqueante o permaneci\'o 
bloqueada durante el \'ultimo ciclo (en cuyo caso proveemos al proceso bloqueado la compensaci\'on 
de tickets correspondiente bas\'andonos en el paper antes mencionado) \'o porque la tarea pid 
termin\'o (en cuyo caso desalojamos la tarea
actual). Despu\'es de hacer estos chequeos, corremos $run\_lottery$ que se encarga de efectuar el 
sorteo de tickets para la pr\'oxima vez que se necesite elegir un proceso ganador y as\'i asignarle 
los recursos necesarios. \\
iii) La funci\'on unblock(pid) hace que en la pr\'oxima llamada a la funci\'on tick, el proceso pid 
est\'e disponible para ejecutar. Lo que hacemos, es entregarle al proceso pid la cantidad de tickets
que ten\'ia el proceso antes de ser bloqueado multiplicado por la compensaci\'on correspondiente, que 
depende de qu\'e fracci\'on del quantum dicho proceso hizo del CPU. \\
Por otro lado, la clase SchedLottery tiene una parte privada, que consta de variables como $quantum$, 
que usamos para saber el quantum que tiene disponible cada proceso para ejecutarse, $seed$, que
usamos como semilla para elegir el ticket ganador cada vez que hagamos un sorteo, $vector<ticket>$, que
usamos para saber qu\'e tickets corresponden a cada proceso, $total\_tickets$ para tener r\'apido acceso
a la cantidad de tickets que hay distribuidos entre los procesos y $ticket\_number$ que usamos para 
llevar la cuenta de cu\'antas veces llamamos a la funci\'on tick(cpu,motivo). Adem\'as, contamos con 
las funciones privadas $compensa(pid)$, $desaloja(pid)$ y $tickets\_index(pid)$. La primera se encarga de
calcular y asignar la compensaci\'on deseada a un proceso cuando corresponda; la segunda de que cuando
se bloquea un proceso, no est\'e disponible para ejecutar en la pr\'oxima ejecuci\'on de la funci\'on
tick(cpu,motivo). Es decir, que la probabilidad de ser elegible en el sorteo sobre los tickets sea cero.
Para eso, lo que hicimos fue sacarle todos los tickets temporalmente a dicho proceso. \\ 




\newpage
\index{Parte III - Evaluando los algoritmos de scheduling}
\section{Parte III - Evaluando los algoritmos de scheduling}

\subsection{Ejercicio 6}
\subsection{Ejercicio 7}
\newpage
\subsection{Ejercicio 8}
En esta secci\'on implementamos una segunda versi\'on para el scheduler tipo $Round-Robin$ que ya implementamos
en ejercicios anteriores. En este caso, el objetivo es no permitir la migraci\'on de procesos entre n\'ucleos y
analizar qu\'e sucede con la performance de este scheduler en comparaci\'on con el original para distintos lotes
de tareas. En principio, esperar\'iamos que en algunos tipos de lotes un algoritmo ande mejor que el otro
Esto es porque, por ejemplo, pensamos que si tenemos un 
\'unico proceso que tarda mucho en completarse y no permitimos migraci\'on entre procesos, usando el m\'etodo 
alternativo empeorar\'iamos la performance mientras que usando el otro m\'etodo, el tiempo total que se tardar\'ia
en finalizar el proceso se reducir\'ia a la mitad. Sin embargo, si aparte de dicho proceso largo tuvi\'eramos un
segundo proceso igual de largo, la implementaci\'on m\'as eficiente ser\'ia la que no permite migraci\'on entre 
n\'ucleos (as\'i nos ahorrar\'iamos de perder tiempo en cambios de contexto). Finalmente, en principio creemos que
si hubieran muchos m\'as procesos que n\'ucleos y de longitud similar, 
y no se permitiera la migraci\'on de procesos entre n\'ucleos, los tiempos de finalizaci\'on de distintos 
procesos ser\'ian muy dispares. Algo que, en principio es indeseable pues podr\'ia suceder que si a un proceso muy 
corto le toca justo el mismo core que a un proceso muy largo, este deber\'ia esperar mucho tiempo a que el largo
termine hasta que el procesador lo atienda. \\
En nuestra implementaci\'on del la clase $SchedRR2$ para implementar las mismas tres funciones b\'asicas que venimos
implementando en los anteriores schedulers (tick, unblock, load), utilizamos en la parte privada de la clase tres
estructuras para el manejo de los procesos: estas son un $vector<core>$, que contiene mucha de la informaci\'on 
que ten\'iamos en la implementaci\'on del $SchedRR$(en efecto, tenemos almacenado el quantum del core, la cantidad de
ticks actual, un vector de entero que lleva cuenta de los procesos activos en dicho core, una cola de esos procesos para
saber en qu\'e orden ir ejecut\'andolos, y una funci\'on runNextProcess() que se encarga de sacar de la cola al
primer proceso y ponerlo a correr), pero en este caso efectuamos una distinci\'on entre qu\'e procesos est\'an en cada
n\'ucleo (antes, no nos importaba hacer esta distinci\'on, porque como hab\'ia migraci\'on de procesos entre 
n\'ucleos, no nos importaba en qu\'e n\'ucleo estaba cada proceso). \\
Hacemos uso de esta estructura a trav\'es de la funci\'on $getProcessCores(pid)$, que para 
el proceso pid nos devuelve un puntero al n\'ucleo en el cu\'al est\'a presente. Entonces, ahora
en cada tick del reloj de cada cpu, si el motivo es de bloqueo o terminaci\'on, entonces pasamos a correr en ese mismo
cpu al pr\'oximo proceso. Mientras que si llega un nuevo proceso, lo agregamos a la cola de procesos del cpu
en el cual se efectu\'o el tick y no en ninguno otro. \\
\newpage
\subsection{Ejercicio 9}

\begin{figure}[H]
\centering
\includegraphics[scale=0.66]{../experimentacion/ej9-fairness/tiempo_final/prueba-tiempo-final.png}
\caption{Tiempo de ejecución de cuatro tareas idénticas}
\end{figure}

\begin{figure}[H]
\centering
\includegraphics[scale=0.66]{../experimentacion/ej9-fairness/fairness/plot.png}
\caption{Desvío estándar de la cantidad de ticks asignados a cada proceso}
\end{figure}

\newpage
\subsection{Ejercicio 10}

\begin{figure}[H]
\centering
\includegraphics[scale=0.5]{../experimentacion/ej10-compensation/gant-sin.png}
\caption{Diagrama de $gant$ con $compensation\ tickets$}
\end{figure}

\begin{figure}[H]
\centering
\includegraphics[scale=0.5]{../experimentacion/ej10-compensation/gant-con.png}
\caption{Diagrama de $gant$ sin $compensation\ tickets$}
\end{figure}

\begin{figure}[H]
\centering
\includegraphics[scale=0.66]{../experimentacion/ej10-compensation/plot-comparativa.png}
\caption{Comparación de $fairness$ con y sin $compensation\ tickets$}
\end{figure}

\end{document}
