\begin{center}
  \textbf{Introducción}
\end{center}

    Hoy en día, la globalización y creciente utilización mundial masiva de medios de
    información como la internet, ha impulsado la existencia de  gigantes de la
    información. Ejemplos de estos son empresas como Facebook, Yahoo!, Google,
    Twitter y otros. La mayor parte del servicio provisto por estas empresas a sus
    usuarios, consiste y necesita de la utilización, procesamiento y análisis de
    grandes bases de datos. Por ello, una de las más frecuentes acciones que deben
    ser efectuadas es el filtrado de datos y su posterior procesamiento.

    Un método comúnmente utilizado dentro del entorno de procesamiento de datos a
    través de clusters es el \textbf{MapReduce}, creado por la empresa Google. Este
    permite analizar un gran conjunto de datos, relacionándolos a través de un
    índice (\emph{key}) a través de una operación \texttt{map}, para luego operar
    sobre los conjuntos de valores asociados a cada \emph{key} a través de una
    operación \texttt{reduce}. Opcionalmente, es posible aplicar al conjunto de
    datos producido por \texttt{reduce}, una última operación \texttt{finalize}.

    Considerando la inmensidad del volumen de información con el que se trabaja, y
    el gran costo económico que implican la fabricación y el funcionamiento de los
    servidores adecuados para manipularla, es importante lograr el máximo
    aprovechamiento de los recursos computacionales de los que estas empresas
    disponen. Por ello, se intenta trabajar con la máxima eficiencia algorítmica,
    temporal y de espacio posibles.
