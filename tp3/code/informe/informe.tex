% ------ headers globales -------------
\documentclass[11pt, a4paper, twoside]{article}
\usepackage{header}
\usepackage{config}
% -------------------------------------
\begin{document}

% -- Carátula --
\clearpage{\pagestyle{empty}% parametros para la caratula (caratula.sty)

\materia{Sistemas Operativos}
%\submateria{}
\titulo{Trabajo Práctico 3}
\subtitulo{Algoritmos en sistemas distribuidos}
%\subtitulo{Escape en Sistemas}
\fecha{13 de noviembre de 2014}
\integrante{Rodriguez, Pedro}{197/12}{pedro3110.jim@gmail.com}
\integrante{Benegas, Gonzalo Segundo}{958/12}{gsbenegas@gmail.com}
\integrante{Barrios, Leandro Ezequiel}{404/11}{ezequiel.barrios@gmail.com}
%\grupo{Grupo ??}

\maketitle
}

%-- Índice --
\clearpage{%
  \pagestyle{empty}\tableofcontents%
  \vspace{3cm}%
  \begin{abstract}
    Hoy en día, la globalización y creciente utilización mundial masiva de medios de
    información como la internet, ha impulsado la existencia de  gigantes de la
    información. Ejemplos de estos son empresas como Facebook, Yahoo!, Google,
    Twitter y otros. La mayor parte del servicio provisto por estas empresas a sus
    usuarios, consiste y necesita de la utilización, procesamiento y análisis de
    grandes bases de datos. Por ello, una de las más frecuentes acciones que deben
    ser efectuadas es el filtrado de datos y su posterior procesamiento.

    Un método comúnmente utilizado dentro del entorno de procesamiento de datos a
    través de clusters es el \textbf{MapReduce}, creado por la empresa Google. Este
    permite analizar un gran conjunto de datos, relacionándolos a través de un
    índice (\emph{key}) a través de una operación \texttt{map}, para luego operar
    sobre los conjuntos de valores asociados a cada \emph{key} a través de una
    operación \texttt{reduce}. Opcionalmente, es posible aplicar al conjunto de
    datos producido por \texttt{reduce}, una última operación \texttt{finalize}.

    Considerando la inmensidad del volumen de información con el que se trabaja, y
    el gran costo económico que implican la fabricación y el funcionamiento de los
    servidores adecuados para manipularla, es importante lograr el máximo
    aprovechamiento de los recursos computacionales de los que estas empresas
    disponen. Por ello, se intenta trabajar con la máxima eficiencia algorítmica,
    temporal y de espacio posibles.
  \end{abstract}
  \cleardoublepage%
}
%-- A partir de aquí, pongo el contador de páginas en 1 --
\setcounter{page}{1}



\index{Implementación de Map-Reduce}
\section{Implementación de Map-Reduce}
\clearpage

\index{Investigación}
\section{Investigación}
Para esta parte del TP se pide leer e interpretar el paper <<\textbf{Job
Scheduling for Multi- User MapReduce
Clusters}>>\footnote{http://www.icsi.berkeley.edu/pubs/techreport
s/ICSI\_jobschedulingfor09.pdf}. Se sugiere además una serie de puntos
para analizar el contenido del paper. Estos son los siguientes:

\vspace{3em}
\ref{investigacion-1} Explique cuál es el entorno y la situación donde se plantea el problema. (Motivación)

\ref{investigacion-2} Identifique de problema

\ref{investigacion-3} Identifique situaciones donde el problema se dispara y la consecuencias que esto genera.

\ref{investigacion-4} Comente el background histórico del problema

\ref{investigacion-5} Explique otros problemas asociados a la búsqueda de una mejor solución

\ref{investigacion-6} Explique las soluciones propuestas

\ref{investigacion-7} Evalúe soluciones propuestas

\ref{investigacion-8} Discusión

\ref{investigacion-9} Conclusiones

\clearpage
\subsection {\footnotesize Explique cuál es el entorno y la situación donde se plantea el problema. (Motivación)}
\label{investigacion-1}

La idea tras este paper surgió luego de que los autores se vieran frente a la
tarea de diseñar un scheduler para MapReduce, para la empresa Facebook. Este
sería el encargado de repartir el poder de cómputo en un \emph{warehouse
Hadoop}\footnote{Hadoop es la versión \emph{open-source} de MapReduce.} de 600
nodos, en un entorno multiusuario, en donde se combinarían tareas de producción
con tareas experimentales.

Este warehouse era utilizado para aproximadamente 3200 operaciones de MapReduce
diarias, siendo que algunas de ellas eran tareas frecuentes de mantenimiento,
análisis de datos, antispam y optimización, cuya ejecución se prolongaba a lo
largo de horas, mientras que otras eran queries \emph{AD-HOC} cuya ejecución
demoraba unos pocos minutos. Por tal motivo, era de vital importancia poder
lograr un mecanismo de \textbf{scheduling justo}.

\clearpage
\subsection {\footnotesize Identifique el problema}
\label{investigacion-2}

Se describen principalmente dos aspectos en donde el scheduling de un cluster
MapReduce se diferencia de un mecanismo genérico de scheduling, y por los cuales
el mecanismo de scheduling utilizado previamente presentaba una pérdida de
rendimiento de entre 2 y 10 veces en comparación con el mecanismo presentado en
el paper. Los dos principales aspectos problemáticos fueron la \emph{localidad
de datos}, y la \emph{interdependencia entre las operaciones Map y Reduce}.

\begin{center}
\textbf{Localidad de Datos}
\end{center}
Este problema comprende la necesidad inherente de la técnica MapReduce de tener
acceso local a los datos con los cuales se están trabajando. Esta necesidad se
debe principalmente a que a diferencia de los algoritmos ``\texttt{CPU-
INTENSIVE}''\footnote{Los que necesitan un gran poder de cómputo.}, el MapReduce
entra en la categoría de los denominados ``\texttt{DATA-INTENSIVE}'', es decir,
aquellos en donde el poder de cómputo necesario es despreciable frente a la gran
cantidad de datos con los que se trabaja.

Desde esta perspectiva, contar con un scheduler eficiente en el acceso y la
utilización de los datos, es mucho más importante que en otro tipo de
situaciones. Todo esto está agravado por el costo que supone el almacenamiento,
y la transmisión de datos entre distintos servidores. Este defecto se puede
observar principalmente en \textbf{trabajos pequeños y/o concurrentes}.


\begin{center}
\textbf{Interdependencia}
\end{center}
Se entiende por interdependencia a la necesidad de que todas las tareas de
\texttt{map} se encuentren finalizadas al momento de iniciar la ejecución de
\texttt{reduce}. Es fácil percibir la gran cantidad de inconvenientes que esto
genera. Se mencionan particularmente los casos de \textbf{infrautilización de
los recursos y \texttt{starvation}}, en donde se da que una única tarea adquiere
cierta cantidad de slots, destinados a realizar la operación \texttt{reduce},
pero que se ven bloqueados/inutilizados hasta el momento en que esta tarea
termine de realizar su operación de \texttt{map}, evitando de esta forma que
otras tareas se ejecuten, y de \textbf{consumo excesivo de espacio de disco}
destinado a los datos intermedios producidos por \texttt{map} (datos que no
pueden ser liberados hasta que el trabajo termine).

\clearpage
\subsection {\footnotesize Identifique situaciones donde el problema se dispara y la consecuencias que esto genera.}
\label{investigacion-3}

\clearpage
\subsection {\footnotesize Comente el background histórico del problema}
\label{investigacion-4}
Hadoop está inspirado en el MapReduce de Google, por lo que gran parte de su
implementación básica está fuertemente ligada a este proyecto. La primer
solución al conflicto del scheduling, se encuentra provista por defecto dentro
la propia implementación de Hadoop, y consta de una cola \texttt{FIFO} de 5
niveles de prioridad, de forma tal que cada vez que un slot se libera, el
scheduler lo asigna a la más prioritaria de entre las tareas pendientes. 

Sobre este enfoque, Hadoop aplica una \texttt{optimización de localidad}, al
igual que lo hace MapReduce de Google: dado que una tarea generalmente consta de
múltiples operaciones de \texttt{map}, luego de elegir una tarea el scheduler
selecciona las operaciones map más adecuadas según un \textbf{criterio de
localidad}. Serán elegidos entonces los maps que utilicen datos que se
encuentren más cerca físicamente al worker que está corriendo la tarea. Es
decir, serán elegidas primero las que se encuentren en ese mismo worker, luego
las que se encuentren en el mismo rack, y finalmente las que se encuentren en
racks remotos.

La gran desventaja del scheduler \texttt{FIFO} es el pésimo tiempo de respuesta
para trabajos pequeños cuando se encuentran intercalados con trabajos grandes.
Por ello, la primer solución que se implementó, es el mecanismo denominado
\textbf{Hadoop On Demand (HOP)}. Este 

\clearpage
\subsection {\footnotesize Explique otros problemas asociados a la búsqueda de una mejor solución}
\label{investigacion-5}

\clearpage
\subsection {\footnotesize Explique las soluciones propuestas}
\label{investigacion-6}

\clearpage
\subsection {\footnotesize Evalúe soluciones propuestas}
\label{investigacion-7}

\clearpage
\subsection {\footnotesize Discusión}
\label{investigacion-8}

\clearpage
\subsection {\footnotesize Conclusiones}
\label{investigacion-9}

\clearpage  

\end{document}
